\begin{verse}
    \section{麻雀}
    \vspace*{2\ccwd}
    \Large\kaishu{
       我是一只喑哑的麻雀

       所以用啄木鸟的方式

       歌唱生活 \@ 快乐与同悲伤

       在树干 \@ 石墙 \@ 玻璃窗上

       刻落斑驳的音符

       零凑的段篇意合出续连的曲章

       \vspace*{2\ccwd}

       清晨 \@ 第一缕阳光

       从眠宿的玉兰枝头醒来

       啄饮白色瓣弧舀存星月的露华

       我欢忻地轻敲每一处丫杈

       (笃笃 \@ 笃笃 \@ ......)

       答谢她温馨的招待

       \vspace*{2\ccwd}

       草地林阴 \@ 独个儿安暇

       天蓝蓝 \@ 黛瓦黄梁

       红窗着爬山虎的绿衣裳

       青苔无心绕檐廊

       我飞向去触吻每一色屋墙

       奏响差可相谐的五调

       \vspace*{2\ccwd}

       余晖脉脉 \@ 竹叶摇摇

       作别昼间嬉闹的玩伴

       也将找寻未可知的归巢

       最后一遭滑跃自顶梢

       我依依地扣响每一茎竹腰

       脆音清扬致没去的晚照

       \vspace*{2\ccwd}

       我是一只无言的麻雀在歌唱

       洋溢快乐或忧伤       
    }
\end{verse}
\newpage